\documentclass{article}
\usepackage{graphicx} % Required for inserting images

\title{Research Methods ASSIGNMENT NO 02 }
\author{Alishba Fatima}
\date{\today}

\begin{document}

\maketitle

\section{Introduction}
\subsection{Git}
Git is a distributed version control system designed to track changes in source code during software development. It enables teams to collaborate efficiently, maintain a detailed history of changes, and manage multiple versions of a project simultaneously. Git is widely used in software development, research, and documentation projects.
\subsection{LaTex}
LaTeX is a typesetting system commonly used for creating structured and professional documents. It excels at producing high-quality outputs for academic papers, reports, and books. The combination of Git and LaTeX is particularly powerful for researchers and developers as it provides a robust framework for collaboration and version control while maintaining document integrity.


\section{Git Workflow Steps}
The Git workflow typically involves the following steps:
\subsection{Initialize a Repository (git init):}
\begin{itemize}
    \item Command: git init
\item This sets up a new Git repository in the current directory.
\end{itemize}

\subsection{Add Changes to Staging Area (git add):}
\begin{itemize}
    \item Command: git add <file>
\item This stages changes to be included in the next commit.
\end{itemize}
\subsection{Commit Changes (git commit):}
\begin{itemize}
    \item Command: git commit -m "Commit message"
\item This saves changes to the repository with a descriptive message.
\end{itemize}
\subsection{Check Repository Status (git status):}
\begin{itemize}
    \item Command: git status
     \item Displays the current state of the repository, including staged and unstaged changes.
\end{itemize}
\subsection{View Commit History (git log):}
\begin{itemize}
    \item Command: git log
\item Shows a detailed history of commits.
\end{itemize}

\subsection{Branching and Merging (git branch, git checkout, git merge):}
\begin{itemize}
    \item Create a branch: git branch <branch-name>
\item Switch branches: git checkout <branch-name>
 \item Merge branches: git merge <branch-name>
\end{itemize}
\subsection{
Push Changes to Remote (git push):}
\begin{itemize}
    \item Command: git push origin main
     \item Uploads local changes to a remote repository.
\end{itemize}





\section{Version Control of LaTeX Documents}

\section{Summary}


\end{document}
